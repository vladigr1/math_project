\documentclass[12pt,twoside]{article}

%\documentclass[a4paper,10pt,twoside]{article}
\usepackage{amsmath,amsfonts,amssymb}
\usepackage{ucs}  % Unicode support
\usepackage{culmus}
%\usepackage{float}
%\usepackage{listings}
\usepackage{color} %red, green, blue, yellow, cyan, magenta, black, white
\definecolor{mygreen}{RGB}{28,172,0} % color values Red, Green, Blue
\definecolor{mylilas}{RGB}{170,55,241}

%\usepackage[cp1255]{inputenc}
\usepackage[utf8x]{inputenc}
\usepackage[english,hebrew]{babel}
%BIDION

%\usepackage{xkeyval}
\usepackage{graphicx}

\usepackage{epstopdf}
%\usepackage{eucal}
%\usepackage{mathrsfs}
%\usepackage{theorem}
%\usepackage{pifont}
\usepackage {epsfig}
%\usepackage[colorlinks=true, bookmarks=true]{hyperref}
\usepackage{bibtopic,wrapfig}
%\usepackage{bbding}
%\usepackage{fancyhdr}
%\usepackage{verbatim}

%\pagestyle{fancy}

%\usepackage{bidi}


%\rhead{\thepage}
%\lfoot{\small \copyright\;\;\; שירה בר-דב, אורט בראודה}
%\rfoot{\thepage}
%\cfoot{}
%\renewcommand{\headrulewidth}{0.4pt}
%\renewcommand{\footrulewidth}{0.4pt}
%\DeclareGraphicsExtensions{.pdf,.png,.jpg}
%\let\arref\ref
%\renewcommand{\ref}[1]{\I{\arref{#1}}}

% User packages
%%%%%%%%%%%%%%%%%%%%%%%%%%%%%%%%
\usepackage{subcaption}



%\rhead{\thepage}
%\lfoot{\small \copyright\;\;\; שירה בר-דב, אורט בראודה}
%\rfoot{\thepage}
%\cfoot{}
%\renewcommand{\headrulewidth}{0.4pt}
%\renewcommand{\footrulewidth}{0.4pt}
%\DeclareGraphicsExtensions{.pdf,.png,.jpg}
%\let\arref\ref
%\renewcommand{\ref}[1]{\I{\arref{#1}}}

\setlength{\parskip}{6pt} \setlength{\parindent}{0pt}
\setlength{\oddsidemargin}{0pt} \setlength{\evensidemargin}{0pt}

% User defined macros
%%%%%%%%%%%%%%%%%%%%%%%%%%%%%%%%

\newtheorem{definition}{הגדרה}[section]
\newtheorem{theorem}{משפט}[section]
\newtheorem{proposition}{טענה}[section]
\newtheorem{conjecture}{השערה}[section]
\newtheorem{corollary}{מסקנה}[section]
\newtheorem{lemma}{למה}[section]
\newtheorem{example}{דוגמה}[section]
\newtheorem{comm}{הערה}[section]

\newcommand{\sumi}[1]{\sum_{#1=1}^n}
\newcommand{\Fn}{{F_2}^n}

%\numberwithin{equation}{section}

%\documentclass{amsart}
%%\usepackage[active]{srcltx} % SRC Specials for DVI Searching
%\usepackage {epsfig}
%% THEOREM Environments ---------------------------------------------------
% \newtheorem{thm}{Theorem}
% \newtheorem{cor}[thm]{Corollary}
% \newtheorem{lemma}[thm]{Lemma}
% \newtheorem{prop}[thm]{Proposition}
% \newtheorem{theorem}[thm]{Theorem}
% \theoremstyle{definition}
% \newtheorem{defn}[thm]{Definition}
% \theoremstyle{remark}
% \newtheorem{rem}[thm]{Remark}
%% MATH -------------------------------------------------------------------
%%%% ----------------------------------------------------------------------
%\setlength{\textheight}{43pc} \setlength{\textwidth}{28pc}
%



\begin{document}

\begin{titlepage}
	
\newcommand{\HRule}{\rule{\linewidth}{0.5mm}} % Defines a new command for the horizontal lines, change thickness here

\center % Center everything on the page

%----------------------------------------------------------------------------------------
%   HEADING SECTIONS
%----------------------------------------------------------------------------------------

\textsc{\LARGE   
% מכללת אורט בראודה
% Name of your university/college
}\\[1.5cm]
\textsc{\LARGE 
המחלקה למתמטיקה שימושית
 % Major heading such as course name
}\\[0.5cm]

%----------------------------------------------------------------------------------------
%   TITLE SECTION
%----------------------------------------------------------------------------------------

\HRule \\[0.4cm]
{ \huge \bfseries
חקירת משחק האורות
% Title of your document
 }\\[0.4cm] 
\HRule \\[1.5cm]

%----------------------------------------------------------------------------------------
%   AUTHOR SECTION
%----------------------------------------------------------------------------------------

\begin{minipage}{0.4\textwidth}
\begin{flushleft} \large
\emph{מאת:}\\
ולדיסלב ברקנס
% Your name
\end{flushleft}
\end{minipage}
~
\begin{minipage}{0.4\textwidth}
\begin{flushright} \large
\emph{מנחה:} \\
אלכס גולוורד 
% Supervisor's Name
\end{flushright}
\end{minipage}\\[2cm]
%----------------------------------------------------------------------------------------
%   DATE SECTION
%----------------------------------------------------------------------------------------

{\large \today}\\[2cm] % Date, change the \today to a set date if you want to be precise
%----------------------------------------------------------------------------------------
%   LOGO SECTION
%----------------------------------------------------------------------------------------
\begin{figure}
	\begin{center}
		\L{\includegraphics[scale=0.3]{images/Braude_Logo.jpg}}
	\end{center}
%	\caption{הפונקציה $\arctan(x)$ - באדום, וסכום שלושת האיברים הראשונים של טור טיילור שלה - בכחול}
%	\label{atan}
\end{figure}

%\includegraphics[scale=0.3]{Braude_Logo}\\[1cm] % Include a department/university logo - this will require the graphics package
%----------------------------------------------------------------------------------------

\vfill % Fill the rest of the page with whitespace

\end{titlepage}
%----------------------------------------------------------------------------------------
%   תוכן עניינים
%----------------------------------------------------------------------------------------
\tableofcontents

\newpage
%--------------------------------------------------------------------------------------
%   הקדמה
%----------------------------------------------------------------------------------------
\section{הקדמה}
% TODO: at the end summary main points

\section{רקע על משחק האורות}
משחק האורות או 
\L{Lights Out}
בלועזית,
זהו משחק בו יש לוח משבצות ריבועי.
\\
כל משבצת הינה לחצן על הלוח שיכולה להיות בשתי מצבים:
דלוק או כבוי.
\\
כאשר לוחצים על משבצת, משבצת הנלחצת וכל משבצות הסמוכות לה כלומר,
כל המשבצת בעל צלע משותפת משנות את מצב נוכחי.
\\
משחק מתחיל כשהלוח כולו עם משבצות דלוקות והמטרה לכבות את כל המשבצות על הלוח כולו.

נתאר זאת ויזואלית: 

\begin{figure}[ht]
    \begin{subfigure}{.5\textwidth}
        \unsethebrew
        \caption{\R{מצב התחלתי}}
        \centering
        \includegraphics{images/4x4_start_board.PNG}
        \sethebrew
    \end{subfigure}%
    \begin{subfigure}{.5\textwidth}
        \unsethebrew
        \caption{\R{לחיצה על משבצת מסומנת}}
        \centering
        \includegraphics{images/4x4_press.PNG}
        \sethebrew
    \end{subfigure}%
\end{figure}

נבחין כי המשחק 
$4 \times 4$
מתחיל במצב
\L{(a)}.
\\
בלוח 
\L{(b)}
נתאר מצב בו לחצו על משבצת המסומנת, בירוק
כל המשבצות השכנות והיא משנות מצבן, היות ומצב של כולן היה דלוקות לכן הן נכבו

המשחק במקור היה צעצוע אלקטרוני על לוח 
$5 \times 5$
ששוחרר ב 
$1995$.
\\
המשחק יכול להראות פשוט אבל כפי שתואר
במאמר
\cite{B1}
\L{"devilish invention"}.
\\
קיים קושי רב בלמצוא שיטה לפתרון אינטואיטיבי, הקושי של משחק מתבלט בשאלה כיצד כדי להתחיל את המשחק?
\\
בנוסף אציין מניסיון האישי שהמשחק קשה כבר 
על לוח 
$5 \times 5$
ולרוב אנשים שמשחקים אותו מכירים מצבים על הלוח שיודע עליהם משם את הפתרון.

פרויקט זה באה בעקבות הקושי של המשחק
והניסוי להציע שיטות לפתרון, בעקבות ניסיונות עלו
נעזרנו במספר רב של כלים מתמטיים מתקדמים.
\\
אחת המטרות במחקר למצוא הסבר לתופעות במשחק שנתקלנו.

נציין כי קיימים עוד המון שאלות שמשחק מעלה ולא לכולם קיים פתרון,
נשמח בפרויקט זה פתרון לכמה מהשאלות שעולות.
\\
חוץ מאתגר של המשחק עצמו קיים אתגר מתמטי שנרצה בפרויקט זה להציג ולהתעניין.


\subsection{ משחק האורות על גרף}
אחרי שכללי המשחק על לוח הובנו אפשר לנסות להכליל את המשחק כמשחק על גרף.
\\
קיימים הרבה סיבות בהם תירצה להגדיר את הבעיה על מבנה כללי שכזה:

\begin{enumerate}
    \item 
    ככול שמבנה כללי יותר תאוריה שאתה מפתח מתאימה ליותר בעיות.
    \item 
    קיימת תאוריה רחבה שפותחה על גרפים ואתכן שנעזר בחלק
    מהטענות מהתאורה שכזה.
    \item 
    מבליט את מהות הבעיה והגדרה הבסיסית ביותר של המשחק.
\end{enumerate}

ארצה להתייחס לנקודה אחרונה, החשיבות הגדולה שאפשר לתאר את הבעיה של משחק
כאוסף של כללים על גרף, מרכזת אותנו לבעיה ובסופו של דבר כשנראה את שיטה למציאת
הפתרון, השיטה עצמה תזכיר לנו מיד את הייצוג הגרפי.

כדי לתאר את משחק האורות על גרף נשתמש באותם כללים שהגדרנו פרט לעובדה
שצמתים הם הלחצנים או המשבצות במקרה של הלוח
וכל לחיצה הופכת את המצב של הצומת והשכנים שלה.
\\
נזכיר כי צמתים שכנים הם צמתים שיש
קשת ביניהם.

נציין כי כאשר כל צומת יכולה להיות בשתי מצבים,
דלוקה או כבויה המטרה היא לעבור מכל הצמתים במצב מסוים דלוק למצב אחר כבוי.
\\
העובדה שמצב התחלתי הינו דלוק או כבוי אינה תשנה את המשחק עלה רק לאיזה מצב סופי צריך לעבור
לכבוי או דלוק.

נמחיש זאת על דוגמה:
\begin{figure}[ht]
    \begin{subfigure}{.5\textwidth}
        \unsethebrew
        \caption{\R{מצב התחלתי}}
        \centering
        \includegraphics[width=\textwidth,height=\textheight,keepaspectratio]{images/graph_start_board.png}
        \sethebrew
    \end{subfigure}%
    \begin{subfigure}{.5\textwidth}
        \unsethebrew
        \caption{\R{לחיצה על משבצת מסומנת}}
        \centering
        \includegraphics[width=\textwidth,height=\textheight,keepaspectratio]{images/graph_press.png}
        \sethebrew
    \end{subfigure}%
\end{figure}

על גרף התחלתי
\L{(a)}
ניתן לראות
$6$
קודקודיים
צבועים באפור ומטרה לצבוע את כולם לצהוב.
\\
בשלב 
\L{(b)}
מציגים  לחיצה על צומת ירוקה היא ושכניה נצבעים בצהוב.

\begin{comm}
    בפועל צומת ירוקה גם נצבעת לצהוב צביעה לירוק נועדה להדגשה על מי נלחץ
\end{comm}

משחק על גרף הינה הכללה  של משחק על לוח כלומר, כל משחק 
לוח ניתן לתאר בעזרת משחק על גרף.

נמחיש זאת על דוגמה:

ניקח לוח למשל
$2 \times 3$
נמספר את המשבצות כמו באיור
\ref{2x3_board}

\begin{figure}[ht]
    \caption{
        \R{לוח
        $2 \times 3$
        }
    }
    \unsethebrew
    \centering
    \label{2x3_board}
    \includegraphics[width=0.5\textwidth,height=0.5\textheight,keepaspectratio]{images/2x3_board.PNG}
    \sethebrew
\end{figure}

כדי לתאר את הלוח על על משחק גרף נשתמש בשני הכללים הבאים:

\begin{enumerate}
    \item 
    כל משבצת על משחק לוח נהפוך לנקודה.
    \item 
    כל זוג משבצות סמוכות על לוח נחבר את נקודות בקו
\end{enumerate}

הגרף שנקבל עבור לוח באיור
\ref{2x3_board}
מתואר באיור
\ref{2x3_graph}

\begin{figure}[ht]
    \caption{
        \R{גרף
        $2 \times 3$
        }
    }
    \unsethebrew
    \centering
    \label{2x3_graph}
    \includegraphics[width=0.5\textwidth,height=0.5\textheight,keepaspectratio]{images/2x3_graph.png}
    \sethebrew
\end{figure}

נשים לב שלגרף בו יש צומת אם יותר מ
$4$
שכנים לא ניתן לתאר לוח שכזה כיוון שלכל היות במשחק על לוח
לכל משבצת יש לכל יותר 
$4$
משבצות סמוכות.

לכן לסיכום אפשר למדל כל משחק לוח על משחק גרף שמייצג אותו אבל 
ההפך זה לא נכון.

\newpage

\section{ אלגוריתם למציאת פתרון}
לפני שנציעה לעלות פתרון, נשאל את עצמנו מדוע בכלל צריך למצוא פתרון.
הרי בסופו של דבר זה משחק ולהציע פתרון למשחק יפגע במהותו משחק הרי אף אחד לא ירצה
לשחק במשהו שידוע מה הפתרון שלו.

הצורך למצוא פתרון הוא נוראה טבעי וזה בעקבות שמששחק עצמו מעניין, כשאתה מתחיל 
את המשחק על לוח 
$3 \times 3$
המשחק ניראה תמים ופשוט אתה מתחיל לצפות לאיזושהי חוקיות.
\\
בשלב הזה שאתה כבר מנסה לוח 
$4 \times 4$
המשחק מתגלה כלא פשוט כשאתה מנסה לוח בקונפיגורציה כלשהי לא בהכרח התחלתית
מהר מאד אתה נעבד.
\\
בשלב מסוים גם לוח 
$4 \times 4$
נהיה מוכר ובאופן תמים תנסה לעבור ללוח
$5 \times 5$
ומהר מאד הלוח שובר את רוחה.
קיימים כל כך הרבה מכירים שנשארת לך משבצת אחת שנותרה לסדר ואינה נעלמת
האינטואיציה שחשבת שפיתח על לוח 
$4 \times 4$
נעלמת כאילו למדת לשחק משחק חדש לגמרי.

התופעה הזאת ששינוי גודל מרגיש שהתחלת משחק אחר עוד
מורגשת בשלב שאתה מנסה לפתור את מצב התחלה בלוחות שונים


\begin{figure}[ht]
    \caption{\R{פתרונות של משחק על לוחות שונים}}
    \unsethebrew
    \label{fig:sol_3_4_5}
    \centering
    \begin{subfigure}[b]{.25\linewidth}
    \includegraphics[width=\linewidth]{images/3x3_sol.PNG}
    \end{subfigure}
    \begin{subfigure}[b]{.25\linewidth}
    \includegraphics[width=\linewidth]{images/4x4_sol.PNG}
    \end{subfigure}
    \begin{subfigure}[b]{.25\linewidth}
    \includegraphics[width=\linewidth]{images/5x5_sol.PNG}
    \end{subfigure}

\end{figure}
\sethebrew

איור
\ref{fig:sol_3_4_5}
באה להמחיש את חוסר  אינטואיציה
כאשר האיור מתאר את פתרון של משחק על הלוח כאשר הלוח במצב התחלתי בו כל נורות דלוקות.
כדי שהשחקן ינצח עליו ללחוץ על המשבצות הירוקות.
\\
איור באה להראות שלוחות על קטנים מ
$5 \times 5$
אתכן ותחשוב שפתרון נוצר על לחיצות סימטריות והאיור ממשיך שזה לא כך 
כי כאשר מסתכלים על הלוח 
$5 \times 5$
מיד אפשר לראות שפתרון לא ניראה סימטרי.

חוסר האינטואיציה מתבלט גם מהעבודה שכמות הפתרונות משתנה לכל לוח.
עבור לוח 
$3 \times 3$
קיים פתרון יחיד,
אבל ללוח 
$4 \times 4$
קיים
$16$
פתרונות.
כמה פתרונות היה ללוח
$5 \times 5$
,
האם זה יותר או פחות מלוח
$4 \times 4$
בהפתעה רבה ללוח 
$5 \times 5$
יש רק 
$4$
פתרונות שזה  מפתיע כי אפשר היה לצפות שמספר פתרונות על לוח גדול יותר אגדל.

אפשר להוסיף שעבור לוחות ריבועים כלומר
$n x n$
כמות הפתרונות כל כך לא צפויה כי עברו 
$n \in [1,20]$
מספר הפתרונות הגדול ביותר הוא ללוח
$19x19$
ומספר פתרונות 
הוא 
$65536$.
מספר הפתרונות השני הגדול ביות הוא רק
$256$.

חוץ מבעיית חוסר אינטואיציה לחיפוש פתרון  טבעי אפשרי לנסות
פתרון נאיבי המנסה כל לחיצה .
\\
פתרון הנאיבי נפסל ברגע הזה שחושבים על כמה קומבינציות לחיצה קיימות.

\begin{lemma}
    כמות האפשרויות לחיצה על לוח
    $m \times n$
    הוא 
    $2^{m \cdot n}$
\end{lemma}
נומר שאפשרויות לחיצה זה חסם על כמות המצבים האפשריים שמשחק יוכל להיות.
חסימה זאת נובעת משאלה אם שחקן לוחץ על לחצן כמה יכול להשפיע על הלוח.
מובן שאם שחקן לחץ על לחצן מספר זוגי של פעמיים המצב יחזור למצב שהיה.
לכן,
כל לחצן משפיע על הלוח אם הוא נלחץ או לא כלומר, יכול להיות בשתי מצבים.
היות וללוח
$m \times n$
קיים 
$m \cdot n$
לחצנים
,
היות וכל לחצן 
יכול להיות בשתי מצבים שונים
לכן נקבל 
שמספר אפשרויות לחיצה 
ללחצן 
$2$
ול
$m x n$
לחצנים
$2^{m \cdot n}$.

כבר בלוח 
$6x6$
כמות  אפשרויות לחיצה גדולה 
מכמות הסטנדרטית שמציגים מספר שלמים,
4 בתים או 
$2^32$
מספרים,
המטרה של המחשה זה להדגיש כמה לא פרקטית אופציית הפתרון שכזה.

עכשיו שיש לנו מוטיבציה למצוא פתרון השאלה היא באיזה כלים בעבודה זה נציע להסתכל על שיטה
הממדל את הבעיה לשדה לינארי ולעזר בכלים של אלגברה לינארית למצוא מספר פתרונות שונים.

% TODO: תאוריה ל שדות סופיים

אתכן ויש כמה דרכים להגיע לאותה מודל לינארי שנציע, נציג בעובדה זה שני דרכים אחת 
בעזרת וקטורי שינוי ומניסיון למצוא צירוף לינארי של וקטורים עלו נמצא את פתרון, דרך שניה תהיה
לפי מערכת משוואות שמתארת את הבעיה.
\\
בשני הדרכים נראה שהגנו לאותה מערכת משוואות ונפתור את המשחק בעזרת חיפוש פתרון של המערכת.
\\
בחרנו להציג קודם בעזרת וקטורי שינוי משום שהגדרת וקטורים פשוטה יותר להסבר לאחר שניראה את הדרך הראשונה
הדרך השנייה קלה יותר להסבר.


כדי למדל את הבעיה על שדה לינארי נזכר בייצוג גרפי שאומר כי לחיצה על צומת משנה את הצומת ושכניה 
אם נסמן את צמתים ב
$n_i$
אז אפשר לתאר כי המצב אתחלתי של משחק על גרף הוא שכל צומת אם הערך התחלתי
$n_i = 0$
וכל צומת יכול לקבל 2 ערכים שנסמן אותם ב
${0,1}$
כאשר 
$0$
מצב התחלתי שכל צמתים התחילו 
ו
$1$
מצב סופי של משחק 
המשחק מסתיים כשכל הצמתים מקיימים
$n_i = 1$.
לכן אנחנו עובדים על שדה בינארי.

אפשר לתאר לחיצה על צומת 
$i$
כווקטור שינוי ערכי צמתי שכנים

לדוגמה ניקח 
משחק בגודל
$2x2$
נמספר את הצמתים 
שורות ואז עמודות מלמעלה למטה כלומר כמו מתואר באיור
\ref{fig:numbering_board_2x2}

\begin{figure}[ht]
    \caption{\R{מספור לוח}}
    \label{fig:numbering_board_2x2}
    \unsethebrew
    \centering
    \includegraphics[width=.5\textwidth,height=.5\textheight,keepaspectratio]{images/numbering_board_2x2.PNG}
\end{figure}
\sethebrew

\begin{comm}
    מספור לפי שורה עליונה על כל עמודות עד לשורה תחתונה תהיה שיטת המספור לאורך כל ספר
\end{comm}

לאחר מספור שכזה נוכל לומר שלחיצה על משבצת 
$1$
וקטור שינוי שלה היה
$
    t_1 = 
    \begin{bmatrix}
        1 \\
        1 \\
        0 \\
        1 \\
    \end{bmatrix}
$
\\
$t_1$
מתאר את עלו צמתים יכול שינוי

עבור גרף וקטור שינוי הי כדומה.
\\
עבור גרף באיור 
מתקבל וקטור שינוי של צומת 
$1$
כך
\ref{fig:numbering_graph}
$
    t_1 = 
    \begin{bmatrix}
        1 \\
        0 \\
        1 \\
        1 \\
    \end{bmatrix}
$

\begin{figure}[ht]
    \caption{\R{גרף ממוספר}}
    \label{fig:numbering_graph}
    \unsethebrew
    \centering
    \includegraphics[width=.7\textwidth,height=.7\textheight,keepaspectratio]{images/numbering_graph.PNG}
\end{figure}
\sethebrew

\begin{definition}
    וקטור השינוי
    שנסמן ב
    $\vec{t_j}$
    של לחצן
    ממוספר
    $i$.
    \\
    וקטור
    שייך 
    לשדה 
    $\Fn$
    כאשר 
    $F_2$
    שדה בינארי
    ו 
    $n$
    הינו מספר הלחצנים.
    נגדיר שערכיו
    $t_{i,j} = 1$
    עבור
    לחצנים
    $j$
    שכנים עליו ועל עצמו
    ושאר ערכי וקטור שווים ל
    $0$.
\end{definition}

\begin{comm}
    היות ווקטור שינוי שדה
    $\Fn$
    חיבור בין וקטורים הינו חיבור בין האינדקסים מודול 
    $2$
    וכפל בסקלר
    הוא לכפול את כל ערכי וקטור בסקלר
    כאשר הסקלרים יכולים להיות
    $0$
    או 
    $1$
\end{comm}

נזכיר שלחצנים שכנים הם לחצנים שמשתנים אתו באת לחיצה 
אם זה במקרה הגרפי צמתים שכנים כלומר בעלי צומת משותפת
או במקרה הלוח זה לחצנים בעלי צלע משותפת.

בעזרת וקטורי השינוי אפשר לתאר תוצאה של קומבינציות של לחיצות
בעזרת צירוף לינארי של וקטורי שינויים.
את המצב המתקבל נוסיף למצב הקיים ונקבל את השינוי שנוצר בלחיצה של כפתורים.
נדגים רעיון זה על איור
\ref{fig:start graph presses}
.

נניח שצומת 1 היא יחידה שדלוקה.
\\
נרצה להראות איך הגרף יראה עם ילחצו על כפותרים 
$1, 3$.

\begin{figure}[ht]
    \caption{\R{מצב התחלתי}}
    \label{fig:start graph presses}
    \unsethebrew
    \centering
    \includegraphics[width=.7\textwidth,height=.7\textheight,keepaspectratio]{images/graph_presses.png}
\end{figure}
\sethebrew

$
    t_1 + t_3 = 
    \begin{bmatrix}
        1 \\
        0 \\
        1 \\
        1 \\
    \end{bmatrix}
    +
    \begin{bmatrix}
        1 \\
        1 \\
        1 \\
        1 \\
    \end{bmatrix}
    =
    \begin{bmatrix}
        0 \\
        1 \\
        0 \\
        0 \\
    \end{bmatrix}
$

ומצב התחלתי שמתואר באיור נסמן ב
$S_0$
לכן מתקבל

$
    S_0 + t_1 + t_3 = 
    \begin{bmatrix}
        1 \\
        0 \\
        0 \\
        0 \\
    \end{bmatrix}
    +
    \begin{bmatrix}
        0 \\
        1 \\
        0 \\
        0 \\
    \end{bmatrix}
    =
    \begin{bmatrix}
        1 \\
        1 \\
        0 \\
        0 \\
    \end{bmatrix}
$

וקטור התוצאה שהתקבל אכן תואם לתוצאה המצופה
מתואר באיור 
\ref{fig:start graph presses solution}.

נשים לב שעבור איך שהגדרנו את המשחק 
\L{Light out}
המשחק מתחיל כשכול
נורות דלוקות או כבויות
ומטרה היא לכבות או להדליק את כל נורות.
\\
כפי שאמרנו שני המשחקים שקולים ההבדל רק
מה המשמעות שנותנים לערכים
$0 ,1 $ 
ומהו מצב הלוח התחלתי בהגדרה זה.

\begin{figure}[ht]
    \caption{\R{מצב לאחר ביצוע הלחיצות}}
    \label{fig:start graph presses solution}
    \unsethebrew
    \centering
    \includegraphics[width=.7\textwidth,height=.7\textheight,keepaspectratio]{images/graph_presses_solve.png}
\end{figure}
\sethebrew

מפרק זה נגדיר שאנחנו פותרים את המשחק שלוח התחלתי כולו באפסים
ומטרה להגיע ללוח שכולו אחדים.
תיאור הטבעי למשחק שכזה להתחיל מלוח כבוי ומטרה להדליק את כל הנורות

מתיאור שכזה מובן כי 
$S_0$
זהו וקטור אפסים לכן כדי לתאר את ממצב התחלתי
למצב לצירוף לינארי של וקטורי שינוי 
אין צורך לחבר בין מצב התחלתי וצירוף לינארי.

היות ומתקיים:

\begin{equation}
    \label{eq: sum change vectors}
    S_0 + \sumi{j} a_i \vec{t_j} =  \sumi{j} a_j \vec{t_j}
\end{equation}

בעקבות כך ניתן לתאר את בעיית המשחק לצורה הבאה:

\begin{equation}
    \label{eq: lin eq for solving problem}
    \sumi{j} a_j \vec{t_j} = \vec{1}
\end{equation}

כאשר
$\vec{1}$
וקטור שכל ערכיו אחדים
ו
$n$
מספר הצמתים בגרף.

תיאור שכזה מדגיש מספר תכונות 

\begin{lemma}
    סדר הליחצות לא משנה את התוצאה הסופית
\end{lemma}
נשים לב שאם ידוע קבוצת לחיצות ממצב התחלתי הערך של משבצת מסוימת נקבע לפי 
הערך של וקטור תוצאה
בנוסחה
\ref{eq: sum change vectors}.
נשים לב שתוצאת הסכום איננה תלויה בסדר הלחיצות של הוקטור. 

מערכת משוואות 
שמתוארת בנוסחה
\ref{eq: lin eq for solving problem}
אפשר לתאר במספר צורות ונפוצה מבינהם היא
בעזרת מטריצה כמו שמתואר 
בנוסחה 
\ref{eq: matrix eq for solving problem}.

\begin{equation*}
    \begin{bmatrix}
        t_1 & t_2 & \cdots & t_n
    \end{bmatrix}
    \begin{bmatrix}
        a_1 \\
        a_2 \\
        \cdots \\
        a_n \\
    \end{bmatrix}
    =
    \begin{bmatrix}
        1 \\
        1 \\
        1 \\
        1 \\
    \end{bmatrix}
\end{equation*}
\begin{equation}
    \label{eq: matrix eq for solving problem}
    \begin{bmatrix}
        t_{1,1} & t_{1,2} & \cdots & t_{1,n} \\
        t_{2,1} & t_{2,2} & \cdots & t_{2,n} \\
        \cdots & \cdots & \cdots & \cdots\\
        t_{i,j} & t_{i,2} & \cdots & t_{i,n} \\
        \cdots & \cdots & \cdots & \cdots\\
        t_{n,1} & t_{n,2} & \cdots & t_{n,n} \\
    \end{bmatrix}
    \begin{bmatrix}
        a_1 \\
        a_2 \\
        \cdots \\
        a_n \\
    \end{bmatrix}
    = 
    \begin{bmatrix}
        1 \\
        1 \\
        \cdots \\
        1 \\
    \end{bmatrix}
\end{equation}

תוצאה דומה אפשר לראות בפרק מהספר
\cite{B2}.
\\
מרגע שהצלחנו לתאר את הבעיה מערכת משוואות לינארית
על שדה
$\Fn$
מכן נוכל להעזר בכלים של אלגברה לינארית כדי למצוא את הפתרון כמו מציאת פתרון בעזרת דירוג,
מציאת מטריצה פסאדו הפוכה וכולי.

\begin{comm}
    \label{comm: for board too many variables}
    עבור לוח 
$[n \times n]$
כמות הלחצנים 
$n^2$
ולכן גודל המטריצה שתוארה במשוואה
\ref{eq: matrix eq for solving problem}
היא 
$[n^2 \times n^2]$
\end{comm}


כלומר כמות הזיכרון שצריך כדי לשמור מטריצה ללוח ריבועי שאורך צלע הוא
$n$
דורש לשמור מטריצה עם 
$n^4$
ערכים
.


\subsection{פתרון בעזרת שיטה הספרדית}
הצגנו גישה פתרון בעזרת תיאור המערכת בעזרת וקטורים וחיפוש פתרון על ידי פתירת מערכת משוואת לינאריות.
כפי שתואר בעבודה 
\cite{B2}
החידה הוצגה כצעצוע ומתאימה למודל שקראנו לו משחק על לוח ריבועי.
לפי הערה 
\ref{comm: for board too many variables}
כמות המידע שצריך לשמור גדל בקצב 
$n^4$
כאשר
$n$
כמות הלחצנים לשורה.
\\
במאמר 
\cite{B1}
מציג שיטה שמציאה פתרון עם מערכת משוואות לינארית אחרת שמובילה גם היא לפתרון.
מערכת משוואת לינארית שמאמר 
\cite{B1}
מתאר ממדי
$n \times n $.
כלומר כמות הערכים המטריצה המתקבלת שווה
ל
$n^2$
שקטנה בריבוע
ממערכת 
שנתונה במשוואה
\ref{eq: matrix eq for solving problem}.

בפרק זה נציג את הגישה שמתוארת 
\cite{B1}
נתאר כמה הבחנות שיסתמכו ששני הגישות שקולות ושגישה החדשה הינה רק אופטימיזציה
ספציפית לצורה של מטריצה הנתונה.
את הגישה החדשה ניקרא לאורך כל הפרק הגישה הספרדית.

נתאר את הגישה הספרדית בעזרת דוגמה על לוח 
$3 \times 3$
זה הינה גישה כללית ללוח משחק בגודל 
$n$
כלשהו פשוט מקל על ניסוח והבנה.

נניח שאיור 
\ref{fig: 3 x 3 board indexed}
מתאר את הלוח הנתון כאשר המשבצות הינן הלחצנים ומספור זה אינדקסים השם 
של הלחצן.

\begin{figure}[ht]
    \caption{לוח 
    $3 \times 3$
    עם אינדקסים}
    \label{fig: 3 x 3 board indexed}
    \unsethebrew
    \centering
    \includegraphics[width=.3\textwidth,height=.3\textheight,keepaspectratio]{images/3x3_board_index.PNG}
\end{figure}
\sethebrew

שיטת הפתרון של מאמר הספרדי הינה להתחל את הלוח עם משתנים ולמלאות את הלוח במשתנים עלו.
לאחר שכל הלוח מלא מתקבלים 
$n$
משוואות במקרה של דוגמה 
$3$
וכל משוואה תהיה עם 
$n$
נעלמים
ולכן נקבל מערכת משוואת שמטריצה המייצגת הינה מסדר 
$n \times n$

שיטת הספרדית מתחיל בכך
שממלאים את השורה העליונה במשתנים
כפי שמתואר באיור 
\ref{fig: 3 x 3 board init spanish}

\begin{figure}[ht]
    \caption{לוח 
    $3 \times 3$
    מאותחל}
    \label{fig: 3 x 3 board init spanish}
    \unsethebrew
    \centering
    \includegraphics[width=.3\textwidth,height=.3\textheight,keepaspectratio]{images/3x3_first_row.PNG}
\end{figure}
\sethebrew

\section{הוכחת  קיום פתרון עבור כל גרף}
% TODO:

\subsection{מספר הפתרונות עבור כל גרף}
% TODO:

\section{פתרון מינימלי עבור לוחות מלבניים}
% TODO:

% TODO: speak about solution on non stating board

\section{תוצאות ומסקנות}
% TODO:

\section{נספחים}
% TODO:

%----------------------------------------------------------------------------------------
%   רשימת מקורות
%----------------------------------------------------------------------------------------

\begin{thebibliography}{99}
\unsethebrew
\bibitem{B1} ALL LIGHTS AND LIGHTS OUT
An investigation among lights and shadows by
SUMA magazine’s article by Rafael Losada
Translated from Spanish by Ángeles Vallejo

\bibitem{B2} Lecture 24: Light out Puzzle , SFU faculty of scienc department of mathematics
\end{thebibliography}

\end{document} 
