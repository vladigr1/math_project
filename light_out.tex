\documentclass[12pt,twoside]{article}

%\documentclass[a4paper,10pt,twoside]{article}
\usepackage{amsmath,amsfonts,amssymb}
\usepackage{ucs}  % Unicode support
\usepackage[utf8x]{inputenc}
\usepackage{culmus}
%\usepackage{float}
%\usepackage{listings}
\usepackage{color} %red, green, blue, yellow, cyan, magenta, black, white
\definecolor{mygreen}{RGB}{28,172,0} % color values Red, Green, Blue
\definecolor{mylilas}{RGB}{170,55,241}

%\usepackage[cp1255]{inputenc}
\usepackage[english,hebrew]{babel}
%BIDION

%\usepackage{xkeyval}
\usepackage{graphicx}

\usepackage{epstopdf}
%\usepackage{eucal}
%\usepackage{mathrsfs}
%\usepackage{theorem}
%\usepackage{pifont}
\usepackage {epsfig}
%\usepackage[colorlinks=true, bookmarks=true]{hyperref}
\usepackage{bibtopic,wrapfig}
%\usepackage{bbding}
%\usepackage{fancyhdr}
%\usepackage{verbatim}

%\pagestyle{fancy}

%\usepackage{bidi}


%\rhead{\thepage}
%\lfoot{\small \copyright\;\;\; שירה בר-דב, אורט בראודה}
%\rfoot{\thepage}
%\cfoot{}
%\renewcommand{\headrulewidth}{0.4pt}
%\renewcommand{\footrulewidth}{0.4pt}
%\DeclareGraphicsExtensions{.pdf,.png,.jpg}
%\let\arref\ref
%\renewcommand{\ref}[1]{\I{\arref{#1}}}

% User packages
%%%%%%%%%%%%%%%%%%%%%%%%%%%%%%%%
\usepackage{subcaption}



%\rhead{\thepage}
%\lfoot{\small \copyright\;\;\; שירה בר-דב, אורט בראודה}
%\rfoot{\thepage}
%\cfoot{}
%\renewcommand{\headrulewidth}{0.4pt}
%\renewcommand{\footrulewidth}{0.4pt}
%\DeclareGraphicsExtensions{.pdf,.png,.jpg}
%\let\arref\ref
%\renewcommand{\ref}[1]{\I{\arref{#1}}}

\setlength{\parskip}{6pt} \setlength{\parindent}{0pt}
\setlength{\oddsidemargin}{0pt} \setlength{\evensidemargin}{0pt}

% User defined macros
%%%%%%%%%%%%%%%%%%%%%%%%%%%%%%%%

\newtheorem{definition}{הגדרה}[section]
\newtheorem{theorem}{משפט}[section]
\newtheorem{proposition}{טענה}[section]
\newtheorem{conjecture}{השערה}[section]
\newtheorem{corollary}{מסקנה}[section]
\newtheorem{lemma}{למה}[section]
\newtheorem{example}{דוגמה}[section]
\newtheorem{comm}{הערה}[section]

%\numberwithin{equation}{section}

%\documentclass{amsart}
%%\usepackage[active]{srcltx} % SRC Specials for DVI Searching
%\usepackage {epsfig}
%% THEOREM Environments ---------------------------------------------------
% \newtheorem{thm}{Theorem}
% \newtheorem{cor}[thm]{Corollary}
% \newtheorem{lemma}[thm]{Lemma}
% \newtheorem{prop}[thm]{Proposition}
% \newtheorem{theorem}[thm]{Theorem}
% \theoremstyle{definition}
% \newtheorem{defn}[thm]{Definition}
% \theoremstyle{remark}
% \newtheorem{rem}[thm]{Remark}
%% MATH -------------------------------------------------------------------
%%%% ----------------------------------------------------------------------
%\setlength{\textheight}{43pc} \setlength{\textwidth}{28pc}
%

\begin{document}

\begin{titlepage}
	
\newcommand{\HRule}{\rule{\linewidth}{0.5mm}} % Defines a new command for the horizontal lines, change thickness here

\center % Center everything on the page

%----------------------------------------------------------------------------------------
%   HEADING SECTIONS
%----------------------------------------------------------------------------------------

\textsc{\LARGE   
מכללת אורט בראודה
% Name of your university/college
}\\[1.5cm]
\textsc{\LARGE 
המחלקה למתמטיקה שימושית
 % Major heading such as course name
}\\[0.5cm]

%----------------------------------------------------------------------------------------
%   TITLE SECTION
%----------------------------------------------------------------------------------------

\HRule \\[0.4cm]
{ \huge \bfseries
חקירת משחק האורות
% Title of your document
 }\\[0.4cm] 
\HRule \\[1.5cm]

%----------------------------------------------------------------------------------------
%   AUTHOR SECTION
%----------------------------------------------------------------------------------------

\begin{minipage}{0.4\textwidth}
\begin{flushleft} \large
\emph{מאת:}\\
ולדיסלב ברקנס
% Your name
\end{flushleft}
\end{minipage}
~
\begin{minipage}{0.4\textwidth}
\begin{flushright} \large
\emph{מנחה:} \\
אלכס גולוורד 
% Supervisor's Name
\end{flushright}
\end{minipage}\\[2cm]
%----------------------------------------------------------------------------------------
%   DATE SECTION
%----------------------------------------------------------------------------------------

{\large \today}\\[2cm] % Date, change the \today to a set date if you want to be precise
%----------------------------------------------------------------------------------------
%   LOGO SECTION
%----------------------------------------------------------------------------------------
\begin{figure}
	\begin{center}
		%\L{\includegraphics[scale=0.3]{Braude_Logo.eps}}
	\end{center}
%	\caption{הפונקציה $\arctan(x)$ - באדום, וסכום שלושת האיברים הראשונים של טור טיילור שלה - בכחול}
%	\label{atan}
\end{figure}

%\includegraphics[scale=0.3]{Braude_Logo}\\[1cm] % Include a department/university logo - this will require the graphics package
%----------------------------------------------------------------------------------------

\vfill % Fill the rest of the page with whitespace

\end{titlepage}
%----------------------------------------------------------------------------------------
%   תוכן עניינים
%----------------------------------------------------------------------------------------
\tableofcontents

\newpage
%--------------------------------------------------------------------------------------
%   הקדמה
%----------------------------------------------------------------------------------------
\section{הקדמה}
% TODO: at the end summary main points

\section{רקע על משחק האורות}
משחק האורות או 
\L{Lights Out}
בלועזית,
זהו משחק בו יש לוח משבצות ריבועי
וכל משבצת הינה לחצן על הלוח.
\\
כל משבצת יכולה להיות בשתי מצבים:
דלוק או כבוי.
\\
כאשר לוחצים על משבצת, משבצת הנלחצת וכל משבצות הסמוכות לה כלומר,
כל המשבצת בעל צלע משותפת משנות את מצב נוכחי.
\\
משחק מתחיל כשהלוח כולו עם משבצות דלוקות והמטרה לכבות את כל המשבצות על הלוח כולו.

נתאר זאת ויזואלית: 

\begin{figure}[ht]
    \begin{subfigure}{.5\textwidth}
        \unsethebrew
        \caption{\R{מצב התחלתי}}
        \centering
        \includegraphics{images/4x4_start_board.PNG}
        \sethebrew
    \end{subfigure}%
    \begin{subfigure}{.5\textwidth}
        \unsethebrew
        \caption{\R{לחיצה על משבצת מסומנת}}
        \centering
        \includegraphics{images/4x4_press.PNG}
        \sethebrew
    \end{subfigure}%
\end{figure}

נבחין כי המשחק 
$4x4$
מתחיל במצב
\L{(a)}.
\\
בלוח 
\L{(b)}
נתאר מצב בו לחצו על משבצת המסומנת, בירוק
כל המשבצות השכנות והיא משנות מצבן, היות ומצב של כולן היה דלוקות לכן הן נכבו

המשחק במקור היה צעצוע אלקטרוני על לוח 
$5x5$
ששוחרר ב 
$1995$.
\\
המשחק יכול להראות פשוט אבל כפי שתואר
במאמר
\cite{B1}
\L{"devilish invention"}.
\\
קיים קושי רב בלמצוא שיטה לפתרון אינטואיטיבי, הקושי של משחק מתבלט בשאלה כיצד כדי להתחיל את המשחק?
\\
בנוסף אציין מניסיון האישי שהמשחק קשה כבר 
על לוח 
$5x5$
ולרוב אנשים שמשחקים אותו מכירים מצבים על הלוח שיודע עליהם משם את הפתרון.

פרויקט זה באה בעקבות הקושי של המשחק
והניסוי להציע שיטות לפתרון, בעקבות ניסיונות עלו
נעזרנו במספר רב של כלים מתמטיים מתקדמים.
\\
אחת המטרות במחקר למצוא הסבר לתופעות במשחק שנתקלנו.

נציין כי קיימים עוד המון שאלות שמשחק מעלה ולא לכולם קיים פתרון,
נשמח בפרויקט זה פתרון לכמה מהשאלות שעולות.
\\
חוץ מאתגר של המשחק עצמו קיים אתגר מתמטי שנרצה בפרויקט זה להציג ולהתעניין.


\subsection{ משחק האורות על גרף}
אחרי שכללי המשחק על לוח הובנו אפשר לנסות להכליל את המשחק כמשחק על גרף.
\\
קיימים הרבה סיבות בהם תירצה להגדיר את הבעיה על מבנה כללי שכזה:

\begin{enumerate}
    \item 
    ככול שמבנה כללי יותר תאוריה שאתה מפתח מתאימה ליותר בעיות.
    \item 
    קיימת תאוריה רחבה שפותחה על גרפים ואתכן שנעזר בחלק
    מהטענות מהתאורה שכזה.
    \item 
    מבליט את מהות הבעיה והגדרה הבסיסית ביותר של המשחק.
\end{enumerate}

ארצה להתייחס לנקודה אחרונה, החשיבות הגדולה שאפשר לתאר את הבעיה של משחק
כאוסף של כללים על גרף, מרכזת אותנו לבעיה ובסופו של דבר כשנראה את שיטה למציאת
הפתרון, השיטה עצמה תזכיר לנו מיד את הייצוג הגרפי.

כדי לתאר את משחק האורות על גרף נשתמש באותם כללים שהגדרנו פרט לעובדה
שצמתים הם הלחצנים או המשבצות במקרה של הלוח
וכל לחיצה הופכת את המצב של הצומת והשכנים שלה.
\\
נזכיר כי צמתים שכנים הם צמתים שיש
קשת ביניהם.

נציין כי כאשר כל צומת יכולה להיות בשתי מצבים,
דלוקה או כבויה המטרה היא לעבור מכל הצמתים במצב מסוים דלוק למצב אחר כבוי.
\\
העובדה שמצב התחלתי הינו דלוק או כבוי אינה תשנה את המשחק עלה רק לאיזה מצב סופי צריך לעבור
לכבוי או דלוק.

נמחיש זאת על דוגמה:
\begin{figure}[ht]
    \begin{subfigure}{.5\textwidth}
        \unsethebrew
        \caption{\R{מצב התחלתי}}
        \centering
        \includegraphics[width=\textwidth,height=\textheight,keepaspectratio]{images/graph_start_board.png}
        \sethebrew
    \end{subfigure}%
    \begin{subfigure}{.5\textwidth}
        \unsethebrew
        \caption{\R{לחיצה על משבצת מסומנת}}
        \centering
        \includegraphics[width=\textwidth,height=\textheight,keepaspectratio]{images/graph_press.png}
        \sethebrew
    \end{subfigure}%
\end{figure}

על גרף התחלתי
\L{(a)}
ניתן לראות
$6$
קודקודיים
צבועים באפור ומטרה לצבוע את כולם לצהוב.
\\
בשלב 
\L{(b)}
מציגים  לחיצה על צומת ירוקה היא ושכניה נצבעים בצהוב.

\begin{comm}
    בפועל צומת ירוקה גם נצבעת לצהוב צביעה לירוק נועדה להדגשה על מי נלחץ
\end{comm}

\newpage

\section{ אלגוריתם למציאת פתרון}
לפני שנציעה לעלות פתרון, נשאל את עצמנו מדוע בכלל צריך למצוא פתרון.
הרי בסופו של דבר זה משחק ולהציע פתרון למשחק יפגע במהותו משחק הרי אף אחד לא ירצה
לשחק במשהו שידוע מה הפתרון שלו.

הצורך למצוא פתרון הוא נוראה טבעי וזה בעקבות שמששחק עצמו מעניין, כשאתה מתחיל 
את המשחק על לוח 
$3x3$
המשחק ניראה תמים ופשוט אתה מתחיל לצפות לאיזושהי חוקיות.
\\
בשלב הזה שאתה כבר מנסה לוח 
$4x4$
המשחק מתגלה כלא פשוט כשאתה מנסה לוח בקונפיגורציה כלשהי לא בהכרח התחלתית
מהר מאד אתה נעבד.
\\
בשלב מסוים גם לוח 
$4x4$
נהיה מוכר ובאופן תמים תנסה לעבור ללוח
$5x5$
ומהר מאד הלוח שובר את רוחה.
קיימים כל כך הרבה מכירים שנשארת לך משבצת אחת שנותרה לסדר ואינה נעלמת
האינטואיציה שחשבת שפיתח על לוח 
$4x4$
נעלמת כאילו למדת לשחק משחק חדש לגמרי.

התופעה הזאת ששינוי גודל מרגיש שהתחלת משחק אחר עוד
מורגשת בשלב שאתה מנסה לפתור את מצב התחלה בלוחות שונים


\begin{figure}[ht]
    \caption{\R{פתרונות של משחק על לוחות שונים}}
    \unsethebrew
    \label{fig:sol_3_4_5}
    \centering
    \begin{subfigure}[b]{.25\linewidth}
    \includegraphics[width=\linewidth]{images/3x3_sol.PNG}
    \end{subfigure}
    \begin{subfigure}[b]{.25\linewidth}
    \includegraphics[width=\linewidth]{images/4x4_sol.PNG}
    \end{subfigure}
    \begin{subfigure}[b]{.25\linewidth}
    \includegraphics[width=\linewidth]{images/5x5_sol.PNG}
    \end{subfigure}

\end{figure}
\sethebrew

איור
\ref{fig:sol_3_4_5}
באה להמחיש את חוסר  אינטואיציה
כאשר האיור מתאר את פתרון של משחק על הלוח כאשר הלוח במצב התחלתי בו כל נורות דלוקות.
כדי שהשחקן ינצח עליו ללחוץ על המשבצות הירוקות.
\\
איור באה להראות שלוחות על קטנים מ
$5x5$
אתכן ותחשוב שפתרון נוצר על לחיצות סימטריות והאיור ממשיך שזה לא כך 
כי כאשר מסתכלים על הלוח 
$5x5$
מיד אפשר לראות שפתרון לא ניראה סימטרי.

חוסר האינטואיציה מתבלט גם מהעבודה שכמות הפתרונות משתנה לכל לוח.
עבור לוח 
$3x3$
קיים פתרון יחיד,
אבל ללוח 
$4x4$
קיים
$16$
פתרונות.
כמה פתרונות היה ללוח
$5x5$
,
האם זה יותר או פחות מלוח
$4x4$
בהפתעה רבה ללוח 
$5x5$
יש רק 
$4$
פתרונות שזה  מפתיע כי אפשר היה לצפות שמספר פתרונות על לוח גדול יותר אגדל.

אפשר להוסיף שעבור לוחות ריבועים כלומר
$n x n$
כמות הפתרונות כל כך לא צפויה כי עברו 
$n \in [1,20]$
מספר הפתרונות הגדול ביותר הוא ללוח
$19x19$
ומספר פתרונות 
הוא 
$65536$.
מספר הפתרונות השני הגדול ביות הוא רק
$256$.

חוץ מבעיית חוסר אינטואיציה לחיפוש פתרון  טבעי אפשרי לנסות
פתרון נאיבי המנסה כל לחיצה .
\\
פתרון הנאיבי נפסל ברגע הזה שחושבים על כמה קומבינציות לחיצה קיימות.

\begin{lemma}
    כמות האפשרויות לחיצה על לוח
    $mxn$
    הוא 
    $2^{m \cdot n}$
\end{lemma}
נומר שאפשרויות לחיצה זה חסם על כמות המצבים האפשריים שמשחק יוכל להיות.
חסימה זאת נובעת משאלה אם שחקן לוחץ על לחצן כמה יכול להשפיע על הלוח.
מובן שאם שחקן לחץ על לחצן מספר זוגי של פעמיים המצב יחזור למצב שהיה.
לכן,
כל לחצן משפיע על הלוח אם הוא נלחץ או לא כלומר, יכול להיות בשתי מצבים.
היות וללוח
$mxn$
קיים 
$m \cdot n$
לחצנים
,
היות וכל לחצן 
יכול להיות בשתי מצבים שונים
לכן נקבל 
שמספר אפשרויות לחיצה 
ללחצן 
$2$
ול
$m x n$
לחצנים
$2^{m \cdot n}$.

כבר בלוח 
$6x6$
כמות  אפשרויות לחיצה גדולה 
מכמות הסטנדרטית שמציגים מספר שלמים,
4 בתים או 
$2^32$
מספרים,
המטרה של המחשה זה להדגיש כמה לא פרקטית אופציית הפתרון שכזה.

עכשיו שיש לנו מוטיבציה למצוא פתרון השאלה היא באיזה כלים בעבודה זה נציע להסתכל על שיטה
הממדל את הבעיה לשדה לינארי ולעזר בכלים של אלגברה לינארית למצוא מספר פתרונות שונים.

% TODO: תאוריה ל שדות סופיים

אתכן ויש כמה דרכים להגיע לאותה מודל לינארי שנציע, נציג בעובדה זה שני דרכים אחת 
בעזרת וקטורי שינוי ומניסיון למצוא צירוף לינארי של וקטורים עלו נמצא את פתרון, דרך שניה תהיה
לפי מערכת משוואות שמתארת את הבעיה.
\\
בשני הדרכים נראה שהגנו לאותה מערכת משוואות ונפתור את המשחק בעזרת חיפוש פתרון של המערכת.
\\
בחרנו להציג קודם בעזרת וקטורי שינוי משום שהגדרת וקטורים פשוטה יותר להסבר לאחר שניראה את הדרך הראשונה
הדרך השנייה קלה יותר להסבר.


כדי למדל את הבעיה על שדה לינארי נזכר בייצוג גרפי שאומר כי לחיצה על צומת משנה את הצומת ושכניה 
אם נסמן את צמתים ב
$n_i$
אז אפשר לתאר כי המצב אתחלתי של משחק על גרף הוא שכל צומת אם הערך התחלתי
$n_i = 0$
וכל צומת יכול לקבל 2 ערכים שנסמן אותם ב
${0,1}$
כאשר 
$0$
מצב התחלתי שכל צמתים התחילו 
ו
$1$
מצב סופי של משחק 
המשחק מסתיים כשכל הצמתים מקיימים
$n_i = 1$.
לכן אנחנו עובדים על שדה בינארי.

אפשר לתאר לחיצה על צומת 
$i$
כווקטור שינוי ערכי צמתי שכנים

לדוגמה ניקח 
משחק בגודל
$2x2$
נמספר את הצמתים 
שורות ואז עמודות מלמעלה למטה כלומר כמו מתואר באיור
\ref{fig:numbering_board_2x2}

\begin{figure}[ht]
    \caption{\R{מספור לוח}}
    \label{fig:numbering_board_2x2}
    \unsethebrew
    \centering
    \includegraphics[width=.5\textwidth,height=.5\textheight,keepaspectratio]{images/numbering_board_2x2.PNG}
\end{figure}
\sethebrew

\begin{comm}
    מספור לפי שורה עליונה על כל עמודות עד לשורה תחתונה תהיה שיטת המספור לאורך כל ספר
\end{comm}

לאחר מספור שכזה נוכל לומר שלחיצה על משבצת 
$1$
וקטור שינוי שלה היה
$
    t_1 = 
    \begin{bmatrix}
        1 \\
        1 \\
        0 \\
        1 \\
    \end{bmatrix}
$
\\
$t_1$
מתאר את עלו צמתים יכול שינוי

עבור גרף וקטור שינוי הי כדומה.
\\
עבור גרף באיור 
מתקבל וקטור שינוי של צומת 
$1$
כך
\ref{fig:numbering_graph}
$
    t_1 = 
    \begin{bmatrix}
        1 \\
        0 \\
        1 \\
        1 \\
    \end{bmatrix}
$

\begin{figure}[ht]
    \caption{\R{גרף ממוספר}}
    \label{fig:numbering_graph}
    \unsethebrew
    \centering
    \includegraphics[width=.7\textwidth,height=.7\textheight,keepaspectratio]{images/numbering_graph.PNG}
\end{figure}
\sethebrew

\begin{definition}
    וקטור השינוי
    שנסמן ב
    $t_i$
    של לחצן
    ממוספר
    $i$
    הינו וקטור בעל ממד
    $[n x 1]$
    כאשר 
    $n$
    הינו מספר הלחצנים,
    וערכיו
    $t_{i,j} = 1$
    עבור לחצנים
    שכנים עליו והוא עצמו
    ושאר ערכי וקטור שווים ל
    $0$.

\end{definition}

נזכיר שלחצנים שכנים הם לחצנים שמשתנים אתו באת לחיצה 
אם זה במקרה הגרפי צמתים שכנים כלומר בעלי צומת משותפת
או במקרה הלוח זה לחצנים בעלי צלע משותפת.

% TODO: explain about vector solution

% TODO: second approach



\subsection{פתרון בעזרת שיטה הספרדית}
% TODO: 

\section{הוכחת  קיום פתרון עבור כל גרף}
% TODO:

\subsection{מספר הפתרונות עבור כל גרף}
% TODO:

\section{פתרון מינימלי עבור לוחות מלבניים}
% TODO:

\section{תוצאות ומסקנות}
% TODO:

\section{נספחים}
% TODO:

%----------------------------------------------------------------------------------------
%   רשימת מקורות
%----------------------------------------------------------------------------------------
\section{} %ביבליוגרפיה
\begin{thebibliography}{99}
\unsethebrew
\bibitem{B1} ALL LIGHTS AND LIGHTS OUT
An investigation among lights and shadows by
SUMA magazine’s article by Rafael Losada
Translated from Spanish by Ángeles Vallejo
\sethebrew

\end{thebibliography}

\end{document} 
